\section{Резульатат работы программы}

Был написан конвертатор для конвертации расписания из XML в JSON, он находиться в файле $task1.py$, результат работы программы в $schedule\_tuesday\_my.json$, время работы конвертера при 100 запусках составило $0.031248807907104492$ секунды.


Далее, с помощью библиотеки $xmltodict$ был создан словарь из данных XML файла и собран в JSON с помощью билблиотеки $json$. Время работы программы при 100 запусках составило $0.04686880111694336$ секунды.


Затем исходный код конвертера был переписан через регулярные выражения. Время работы измененного конвертора при 100 запусках составило $0.04687309265136719$ секунды.


Результат работы всех трех программ не отличается, но заметно отличие во времени работы. Разница между конвертором и программы с билблиотекой объясняется тем, что конвертор был написан под специальную задачу, в то время как библиотека покрывает все случаи. Конвертор с регулярными выражениями работает дольше как раз таки из-за регулярных выражений, потому что требуется поиск по заданному шаблону, когда просто конвертор ищет первое вхождение нужного символа.

Последним был написан конвертор из XML в CSV. Результат вполне ожидаем и единственен засчет формата CSV. Стоит заметить, что возможна двусторонняя конвертация возможна при переводе из XML в JSON и наоборот, но никак не переводе из CSV в исходный файл, поскольку он является по сути данными таблицы.

\clearpage
\section{Исходный код}
Исходный код и результаты работы программы можно найти на\\ \url{https://github.com/pmpknu/cs/tree/main/lab4}
